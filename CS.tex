\documentclass[a4paper]{article}

\usepackage{mypckg2}

\begin{document}

\Titel{Basis pursuit II}{}{}{}

\subsection{4.3 Robustness}

\begin{itemize}
    \item Taking into account the noise in the measurements.
    \item \begin{Def}
    {Robust null space property}{}
    \end{Def}
    \item \begin{Satz}
        {}{}
        Folgen für (\(P_{1,\eta}\)), wenn \(A \) die robust null space property erfüllt.
    \end{Satz}
    \item \begin{Satz}
        {Equivalent characterization of robust null space property}{}
    \end{Satz}
    \item \begin{Def}
    {\(\ell^q\)-robust null space property}{}
    \end{Def}
    \item \begin{Satz}
        {Robustness of the quadratically constrained basis pursuit algorithm}{}
    \end{Satz}
    \item Two remarks
    \item \begin{Satz}
    {}{}Generalization of the previous theorem, using the proof of the previous theorem.
    \end{Satz}
\end{itemize}

\subsection{4.4 Recovery of individual vectors}
    
 \end{document} 
