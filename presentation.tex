\documentclass[11pt]{beamer}
\usetheme{Madrid}
\usefonttheme{serif}

\usepackage[utf8]{inputenc}
\usepackage[english]{babel}
\usepackage[T1]{fontenc}

\usepackage{amsmath}
\usepackage{amsfonts}
\usepackage{amssymb}
\usepackage{graphicx}
\usepackage{tcolorbox}
\tcbuselibrary{theorems, breakable,}
\tcbuselibrary{skins}



\colorlet{myred}{red!80!black}
\colorlet{myblue}{blue!80!black}
\colorlet{mygreen}{green!40!black}
\colorlet{mypurple}{red!50!blue!90!black!80}
\colorlet{mydarkred}{myred!80!black}
\colorlet{mydarkblue}{myblue!80!black}
\colorlet{mymagenta}{magenta}
\colorlet{xcol}{blue!60!black}
\colorlet{Defcol}{red!5}
\colorlet{Satzcol}{blue!90!black!80!cyan!10!white}
\colorlet{Korcol}{blue!10!white}
\colorlet{Axcol}{purple!10!white}
\colorlet{Bemcol}{black!2!white}
\colorlet{Motcol}{cyan!10!white}
\colorlet{Bspcol}{green!50!olive!2!white}
\colorlet{Bewcol}{yellow!5!white}

\newcounter{my}[section]
\def\themytheorem{\thesection.\arabic{my}}

\tcbset{
thmstyle/.style={enhanced, breakable, attach boxed title to top left={yshift=-0.9mm,xshift=2.5mm}, boxrule=0.15mm, boxed title style={boxrule=0.15mm}, fonttitle=\itshape\bfseries},
bewstyle/.style={skin=enhanced, parbox=false,boxrule=0.15mm,leftrule=0.5mm,rightrule=0.15mm, coltitle=black, breakable, fonttitle=\itshape},
bemstyle/.style={skin=enhanced, parbox=false,boxrule=0.15mm,leftrule=0.5mm,rightrule=0.15mm, coltitle=black, breakable, fonttitle=\itshape\bfseries},
}

 \newtcbtheorem[auto counter, number within= section]{Def}{Definition}{thmstyle,colback=Defcol,colframe=red!55!black!98!white,colbacktitle=red!75!black!30!pink}{Def}
 
\newtcbtheorem[use counter from= Def]{Satzz}{Theorem}{thmstyle,colback=Satzcol, colframe=blue!80!black!99!cyan,colbacktitle=blue!85!black!80!cyan!60!white,}{Satz}

\newtcbtheorem[use counter from= Def]{Lemmaa}{Lemma}{thmstyle,colback=Satzcol, colframe=blue!80!black!99!cyan,colbacktitle=blue!85!black!80!cyan!60!white,}{Lemma}

 \newtcbtheorem[use counter from= Def]{Prop}{Proposition}{thmstyle,colback=Satzcol, colframe=blue!80!black!99!cyan,colbacktitle=blue!85!black!80!cyan!60!white,}{Prop}
 
 \newtcbtheorem[use counter from= Def]{Kor}{Corollary}{thmstyle,colback=Korcol, colframe=blue!30!black,colbacktitle=blue!40!white}{Kor}
 
 \newtcbtheorem[use counter from= Def]{Ax}{Axiom}{thmstyle,colback=Axcol, colframe=purple!30!black,colbacktitle=purple!40!white}{Ax}
 
\newtcbtheorem[use counter from= Def]{Bemerkung}{Remark}{bemstyle,colback=Bemcol, colframe=white!30!black,colbacktitle=black!5!white}{Bem}

\newtcbtheorem[use counter from= Def]{Notation}{Notation}{bemstyle,colback=Bemcol, colframe=white!30!black,colbacktitle=white}{Notation}

\newtcbtheorem[use counter from= Def]{Motivation}{Motivation}{thmstyle,colback=Motcol, colframe=cyan!30!black,colbacktitle=cyan!40!white}{Mot}




%\newtcolorbox[use counter from= Def]{Bemerkung}{breakable, empty, coltitle=black, title=\textbf{Bemerkung.}}
%\newtbctheorem[use counter from= Def]{Bemerkung}{Bemerkung}{thmstyle,colback=cyan!10!white, colframe=cyan!30!black,colbacktitle=cyan!40!white}{Bemerkung}



\newtcolorbox{Formel}{bewstyle,colback=blue!90!black!80!cyan!10!white, colframe=blue!30!black,colbacktitle=blue!90!black!80!cyan!10!white,}

% \tcbmaketheorem{Def}{Definition}{thmstyle, colback=red!5,colframe=red!55!black!98!white,colbacktitle=red!75!black!30!pink}{my}{def}

% \tcbmaketheorem{Satz}{Satz}{thmstyle,colback=blue!90!black!80!cyan!10!white, colframe=blue!80!black!99!cyan,colbacktitle=blue!85!black!80!cyan!60!white,}{my}{thm}

% \tcbmaketheorem{Lemma}{Lemma}{thmstyle,colback=blue!90!black!80!cyan!10!white, colframe=blue!80!black!99!cyan,colbacktitle=blue!85!black!80!cyan!60!white,}{my}{lemma}

% \tcbmaketheorem{Prop}{Proposition}{thmstyle,colback=blue!90!black!80!cyan!10!white, colframe=blue!80!black!99!cyan,colbacktitle=blue!85!black!80!cyan!60!white,}{my}{prop}

% \tcbmaketheorem{Kor}{Korollar}{thmstyle,colback=blue!10!white, colframe=blue!30!black,colbacktitle=blue!40!white}{my}{kor}

% \tcbmaketheorem{Beispiel}{Beispiel}{bewstyle,colback=green!10!white!97!olive, colframe=green!65!olive,colbacktitle=green!10!white!97!olive, title=\textbf{Beispiel.}}{my}{bsp}


\DeclareMathOperator{\sen}{sen}
\DeclareMathOperator{\tg}{tg}

\setbeamertemplate{caption}[numbered]

\author[Autor]{Arianna Rast}
\title{Basis pursuit II}
\newcommand{\email}{email}
\newcommand{\CC}{\mathbb{C}}
\renewcommand{\emph}{\textbf}
%\setbeamercovered{transparent} 
\setbeamertemplate{navigation symbols}{} 

\institute[]{LMU Munich} 
\date{\today} 
%\subject{}

% ---------------------------------------------------------
% Selecione um estilo de referência
\bibliographystyle{apalike}

%\bibliographystyle{abbrv}
%\setbeamertemplate{bibliography item}{\insertbiblabel}
% ---------------------------------------------------------

% ---------------------------------------------------------
% Incluir os slides nos quais as referências foram citadas
%\usepackage[brazilian,hyperpageref]{backref}

%\renewcommand{\backrefpagesname}{Citado na(s) página(s):~}
%\renewcommand{\backref}{}
%\renewcommand*{\backrefalt}[4]{
%	\ifcase #1 %
%		Nenhuma citação no texto.%
%	\or
%		Citado na página #2.%
%	\else
%		Citado #1 vezes nas páginas #2.%
%	\fi}%
% ---------------------------------------------------------

\begin{document}

\begin{frame}
\titlepage
\end{frame}

\begin{frame}{Summary}
\tableofcontents 
\end{frame}

\section{Robustness}

\begin{frame}{Robustness}
 How do we handle noise in the measurements in the basis pursuit? I.e. what additional assumptions do we need to obtain similar results, if we consider the problem
 \[\min\left\{\|z\|_1\,\big|\,z\in \CC^N,\;\;\|Az-y\|\le \eta\right\}\,?\tag{$P_{1,\eta}$}\]
 It depends on the norm, in which we measure the noise, i.e. the norm of \(\|Az-y\|\).
\end{frame}


\begin{frame}{Robust null space property}

    At first, we consider the situation where we measure the noise in the \(\ell^1\)-norm, i.e. \(\|Az-y\|=\|Az-y\|_1\).
    \begin{Def}
    {Robust null space property}{} A matrix \(A\in\CC^{m\times N}\) is said to satisfy the \emph{robust null space property} with respect to \(\|\cdot\|\) with the constants \(\rho\in (0,1)\) and \(\tau>0\) relative to a set \(S\subseteq[N]\) iff
    \[\forall v\in \CC^N:\;\; \|v_S\|_1\le \rho\|v_{\overline S}\|_1+\tau \|Av\|\,.\tag{RNSP($\|\cdot\|,\rho,\tau,S$)}\]
    \(A \) satisfies the robust null space property of \emph{order} \(s\) with respect to \(\|\cdot\|\) with the constants \(\rho\in (0,1)\) and \(\tau>0\) RNSP(\(\|\cdot\|,\rho,\tau, s\)) iff \(A \) satisfies RNSP(\(\|\cdot\|,\rho,\tau,S\)) for all sets \(S\subseteq[N]\) with \(|S|\le s\).
    \end{Def}
\end{frame}

\begin{frame}{}
    Motivation dafür: TODO
    \begin{Satzz}
    {Characterization of RNSP}{}
    A matrix \(A\in\CC^{m\times N}\) satisfies RNSP(\(\|\cdot\|,\rho,\tau,S\)) if and only if 
    \[\forall x,z\in \CC^N:\;\; \|z-x\|_1\le \frac{1+\rho}{1-\rho}(\|z\|_1-\|x\|_1+2\|x_S\|)+\frac{2\tau}{1-\rho}\|A(x-z)\|\,.\]
    \end{Satzz}

\end{frame}

\begin{frame}{}
    \begin{Kor}
        {}{}
        Assume that \(A\in\CC^{m\times N}\) satisfies RNSP(\(\|\cdot\|,\rho,\tau,s\)) with \(0<\rho<1\) and \(\tau>0\) and let \(x\in \CC^N\). Then, if
        \[\mathcal L_x:=\left\{x^{\#}\in\CC^N\,\big|\, \|x^{\#}\|_1=\min\{\|z_1\|\,|\,\|Ax-Az\|\le \eta\}\right\}\]
        is the solution set of the problem ($P_{1,\eta}$) with \(y=Ax\), then 
        \[\sup_{x^{\#}\in \mathcal L_x}\|x-x^{\#}\|_1\le \frac{2(1+\rho)}{1-\rho}\sigma_s(x)_1+\frac{4\tau}{1-\rho}\eta\,,\]
        i.e. the solution set \(\mathcal L_x\) is contained in a ball of radius \(\frac{2(1+\rho)}{1-\rho}\sigma_s(x)_1+\frac{4\tau}{1-\rho}\eta\) around \(x\) in the \(\ell^1\)-norm.
    \end{Kor}
\end{frame}

\begin{frame}{Tabelas}
    A Tabela mostra um modelo de tabela.

    \begin{table}[htb]
        \caption{Exemplo de tabela}
        \label{tab:modelo_tabela}
        \centering
        \begin{tabular}{c|c|c|c}
	        \hline
	        \textbf{Pessoa} & \textbf{Idade} & \textbf{Peso} & \textbf{Altura} \\ \hline
	        Marcos & 26    & 68   & 178    \\ \hline
	        Ivone  & 22    & 57   & 162    \\ \hline
	        Sueli  & 40    & 65   & 153    \\ \hline
        \end{tabular}
        
        \medskip
        
        Fonte: Produção do próprio autor.
    \end{table}
\end{frame}

\begin{frame}{Imagens}
    A Figura mostra a logomarca promocional da Universidade Federal do Espírito Santo (UFES).
    
    \begin{figure}[htb]
        \caption{Logomarca Promocional UFES}
        \label{fig:logo_UFES}
        \centering
        %\includegraphics[width=0.5\textwidth]{imagens/logomarca_UFES}
        
        \medskip
        
        Fonte: Produção do próprio autor.
    \end{figure}
\end{frame}

\begin{frame}{Conclusão}
    Texto referente à conclusão.
\end{frame}

\begin{frame}{Trabalhos futuros}
    \begin{itemize}
        \item Estudar $\ldots$
        
        \medskip
        
        \item Explorar $\ldots$
        
        \medskip
        
        \item Analisar $\ldots$
    \end{itemize}
\end{frame}

\section{Referências}
\begin{frame}[allowframebreaks]{Referências}
    \bibliography{referencias}
\end{frame}

\begin{frame}

\begin{center}
    Obrigado!
    
    \email
\end{center}

\begin{figure}[htb]
    \centering
    %\includegraphics[width=0.5\textwidth]{imagens/logomarca_UFES}
\end{figure}

\end{frame}

\end{document}